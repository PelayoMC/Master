\documentclass{ctexart}
\usepackage[spanish]{babel}

\title{Minería de textos} % Sets article title
\author{Pelayo Martínez} % Sets authors name
\date{\today} % Sets date for publication as date compiled

% The preamble ends with the command \begin{document}
\begin{document} % All begin commands must be paired with an end command somewhere

	\maketitle % creates title using information in preamble (title, author, date)
    
	%New section is created
	\section{Tema 1}
    La \textbf{minería de textos} es el proceso de analizar colecciones de textos para descubrir información y patrones que no aparecen de forma explícita en los textos.
    Engloba: 
\begin{itemize}
\item   \textbf{Extracción de información}: consiste en extraer automáticamente información estructurada a partir de textos.
\item   \textbf{Recuperación de información}: consiste en buscar documentos, buscar información dentro de los documentos y en buscar metadatos que describan los documentos. Incluye la búsqueda en todo tipo de repositorios y bases de datos, tanto aisladas como conectadas en red
\item   \textbf{Categorización}: consiste en asignar a un documento una o más categorías en función de su contenido. Las categorías con las que se hace la clasificación están definidas previamente.
\item   \textbf{Agrupamiento de documentos}: es una forma de organización de documentos en grupos en la que ni la naturaleza de los grupos, ni en ocasiones su número están definidos de antemano
\end{itemize}
    
 \begin{flushleft}
La minería de textos se aplica a colecciones de documentos con información textual no estructurada y escrita en lenguaje natural. Esta información normalmente conforma documentos que pueden agruparse en colecciones o corpus.  Se utilizan principalmente técnicas de procesamiento del lenguaje natural y de aprendizaje automático.
\end{flushleft}
\begin{flushleft}
\textbf{Procesamiento del lenguaje natural (PLN)}: rama de la informática cuyo objetivo es el desarrollo de sistemas que permitan a los ordenadores comunicarse con personas utilizando el lenguaje humano. Se utiliza para adquirir conocimientos a partir de cantidades masivas de datos textuales (generar resumenes, sistemas de dialogo, etc). 
\end{flushleft}
Es difícil porque:

\begin{itemize}
\item Alta ambigüedad a todos los niveles (léxico, sintáctico, semántico y de discursos): El entendimiento del lenguaje permite evitar estas ambigüedades.
\item Ciertos aspectos intervienen en la interpretación (saber si se niega la información, si es especulativa, expresiones que hacen referencia a una misma cosa, etc)
\end{itemize}

\end{document} % This is the end of the document